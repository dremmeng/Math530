\documentclass[10pt, oneside]{article} 
\usepackage{amsmath, amsthm, amssymb, calrsfs, wasysym, verbatim, bbm, color, graphics, geometry}
\usepackage{mathrsfs}
\geometry{tmargin=.75in, bmargin=.75in, lmargin=.75in, rmargin = .75in}  

\newcommand{\R}{\mathbb{R}}
\newcommand{\C}{\mathbb{C}}
\newcommand{\Z}{\mathbb{Z}}
\newcommand{\N}{\mathbb{N}}
\newcommand{\Q}{\mathbb{Q}}

\newcommand{\Cdot}{\boldsymbol{\cdot}}

\newtheorem{thm}{Theorem}
\newtheorem{defn}{Definition}
\newtheorem{conv}{Convention}
\newtheorem{rem}{Remark}
\newtheorem{lem}{Lemma}
\newtheorem{cor}{Corollary}
\newtheorem{example}{Example}
\newtheorem{exe}{Exercise}

\title{Homework 1.3: [530]}
\author{[Drew Remmenga]}


\begin{document}
\maketitle
\pagebreak
Exercise 2.28 (a)
\begin{align}
1&=c\frac{1}{2}^{x}, x=1,2,3,4\\
1&=c[\frac{1}{2} +\frac{1}{2}{2}+\frac{1}{2}^{3} + \frac{1}{2}^{4}]\\
1&=c[\frac{1}{2}+\frac{1}{4}+\frac{1}{8}+\frac{1}{16}]\\
1&=c[\frac{8}{16}+\frac{4}{16}+\frac{2}{16}+\frac{1}{16}]\\
1&=c[\frac{15}{16}]\\
c&=\frac{16}{15}\\
\end{align}
\pagebreak

Exercise 2.28 (b)
\begin{align}
F(x) &= \sum_{n=1}^{x} f(n), x = 1,2,3,4\\
F(x)& = \sum_{n=1}^{x} \frac{16}{15}\frac{1}{2}^{n}, x=1,2,3,4
\end{align}
\clearpage
\pagebreak
Exercise 2.32 (a)
\begin{align*}
1&=\int_{0}^{1} .5dx +\int_{1}^{3} .5+c(x-1)dx\\
1&=.5x]_{0}^{1} +.5x-cx+\frac{1}{2}cx^{2}]_{1}^{3}\\
1&=.5+.5(3)-3c+\frac{9}{2}c -.5(1)+c(1)-c\frac{1}{2}(1)^{2}\\
1&=1.5+2c \\
c&=-\frac{1}{4}
\end{align*}
\clearpage
\pagebreak
Exercise 2.32 (b)
\begin{align*}
F(x)= \begin{cases} .5x, 0 <x<1 \\ .5x+\frac{-1}{8}x^{2}+\frac{1}{4}x, 1 \leq x < 3 \\ 0, otherwise \end{cases}
\end{align*}
\clearpage
\pagebreak
Exercise 2.34 (a)
Discrete
\clearpage
\pagebreak
Exercise 2.34 (b)
.4
\clearpage
\pagebreak
Exercise 2.34 (c)
.6
\clearpage
\pagebreak
Exercise 2.35 (a)
\begin{align*}
f(x) = \begin{cases} \frac{1}{N}, x = 1,2,3,...,N \\
0, otherwise 
\end{cases}
\end{align*}
\clearpage
\pagebreak
Exercise 2.35 (b)
\begin{align}
E[X] &= \sum_{x=1}^{N} x * f(x) \\
E[X] &= \frac{1}{N} [1+2+...+N]\\
E[x] &= \frac{1}{N} \frac{N(N+1)}{2} \\
E[x] &= \frac{N+1}{2} \\
E[x^{2}] &= \frac{1}{N} \sum_{x=1}^{x} x^{2} \\
& = \frac{1}{N} \frac{N(N+1)(2N+1)}{6} \\
& = \frac{(N+1)(2N+2)}{6} \\
var[x] & = E[x^{2}] - E^{2} [x] \\
& = \frac{N^{2} -1}{12} \\
\end{align}
\clearpage
\pagebreak
Exercise 2.35 (c)
\begin{align}
E[x] & = 3.5 \\
var[x] & =\frac{35}{12}
\end{align}
\clearpage
\pagebreak
Exercise 2.42 (a)
\begin{align}
1& = \int_{0}^{1} cx(1-x)dx\\
1&= \frac{cx^{2}}{2}-\frac{cx^{3}}{3}]_{0}^{1} \\
1&= c[\frac{1}{2} -\frac{1}{3}] \\
c&= 6
\end{align}
\clearpage
\pagebreak
Exercise 2.42 (b)
\begin{align*}
F(x) = \begin{cases} 3x^{2} -2x^{3}, x \in [0,1) \\
0, otherwise
\end{cases}
\end{align*}
\clearpage
\pagebreak
Exercise 2.42 (c)
\begin{align*}
E[x]  & = \int_{0}^{1} x f(x) dx \\
E[x] &= \frac{1}{2} \\
var[x] & = E[x^{2}] -E^{2} [x] \\
& =  6[\frac{1}{4}-\frac{1}{5}] - \frac{1}{2} ^{2} \\
& = \frac{1}{20} \\
\end{align*}
\clearpage
\pagebreak
Exercise 2.52 (a)
0, .3
1, .33
2, .26
3, .09
4, .02
\pagebreak
Exercise 2.52 (b)
\begin{align*}
f(x) = \begin{cases} .3, x=0 \\
.33, x=1 \\
.26, x=2 \\
.09, x=3 \\
.02, x=4 \\
\end{cases}
\end{align*}
\clearpage
\pagebreak
Exercise 2.52 (c)
E[x] = 1.2
var[x] = 1.06
\clearpage
\pagebreak
Exercise 2.52 (d)
It would cause E[x] and var[x] to decrease.
\clearpage
\pagebreak
Exercise 2.54  (a)
\begin{align*}
f(x) = \begin{cases} .5, x=100 \\
.5, x=250 \\
\end{cases} \\
f(y) = \begin{cases} .25, x=0 \\
.25, x=100 \\
.5, x=200 \\
\end{cases}
\end{align*}
X and ty are not independent.
\clearpage
\pagebreak
Exercise 2.54 (b)
              0 100 200
p(y/x = 100) =.4 =.2 =.4
p(y/x =200) =.1 =.3 =.6
\clearpage
\pagebreak
Exercise 2.54 (c)
cov(x,y) = 1875
\clearpage
\pagebreak
Exercise 2.58 

$p(15 < x < 25) > .2256$
\end{document}