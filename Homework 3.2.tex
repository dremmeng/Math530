\documentclass[10pt, oneside]{article} 
\usepackage{amsmath, amsthm, amssymb, calrsfs, wasysym, verbatim, bbm, color, graphics, geometry}
\usepackage{mathrsfs}
\geometry{tmargin=.75in, bmargin=.75in, lmargin=.75in, rmargin = .75in}  

\newcommand{\R}{\mathbb{R}}
\newcommand{\C}{\mathbb{C}}
\newcommand{\Z}{\mathbb{Z}}
\newcommand{\N}{\mathbb{N}}
\newcommand{\Q}{\mathbb{Q}}

\newcommand{\Cdot}{\boldsymbol{\cdot}}

\newtheorem{thm}{Theorem}
\newtheorem{defn}{Definition}
\newtheorem{conv}{Convention}
\newtheorem{rem}{Remark}
\newtheorem{lem}{Lemma}
\newtheorem{cor}{Corollary}
\newtheorem{example}{Example}
\newtheorem{exe}{Exercise}

\title{Homework 3.2: [530]}
\author{[Drew Remmenga]}


\begin{document}
\maketitle
\pagebreak
Exercise 4.9 (a)
median > mean
\clearpage
Exercise 4.9 (b)
Negative
\clearpage
Exercise 4.9 (c)
Q1 will be further from the median since is is scewed negatively. 
\clearpage
Exercise 4.10 (a)
Right scewed
\clearpage
Exercise 4.10 (b)
Third quartile will be further from the median sinc eit is positively scewed.
\clearpage
Exercise 4.10 (c)
Q1 = 14
Q3 = 33
Q2 = 20
\clearpage
Exercise 4.10 (d)
mode < median < mean becuase its right scewed. 
\clearpage
Exercise 4.10 (e)
Outside value is 62,64,69, IQR == 19, Lowerhinge =-14.5 Upperhinge = 61.5
\end{document}