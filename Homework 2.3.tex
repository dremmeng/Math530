\documentclass[10pt, oneside]{article} 
\usepackage{amsmath, amsthm, amssymb, calrsfs, wasysym, verbatim, bbm, color, graphics, geometry}
\usepackage{mathrsfs}
\geometry{tmargin=.75in, bmargin=.75in, lmargin=.75in, rmargin = .75in}  

\newcommand{\R}{\mathbb{R}}
\newcommand{\C}{\mathbb{C}}
\newcommand{\Z}{\mathbb{Z}}
\newcommand{\N}{\mathbb{N}}
\newcommand{\Q}{\mathbb{Q}}

\newcommand{\Cdot}{\boldsymbol{\cdot}}

\newtheorem{thm}{Theorem}
\newtheorem{defn}{Definition}
\newtheorem{conv}{Convention}
\newtheorem{rem}{Remark}
\newtheorem{lem}{Lemma}
\newtheorem{cor}{Corollary}
\newtheorem{example}{Example}
\newtheorem{exe}{Exercise}

\title{Homework 2.3: [530]}
\author{[Drew Remmenga]}


\begin{document}
\maketitle
\pagebreak
Exercise 3.14 (a)
It is systematic sampleing. A random sample has been selected sequentially based on every nth appearance. 
\clearpage
Exercise 3.14 (b)
It is random sampling as number is chosen by a randomn number. 
\clearpage
Exercise 3.14 (c)
Stratiufied sample by gender.
\clearpage
Exercise 3.18 (a)
Stratified with every list classified by job categories. This is because we see strate hetorgeneity between each classification. 
\clearpage
Exercise 3.18 (b)
Since it is decided to sample 10 of the patients a suiutable sampling method would be systematic sampling. If we select every fith patient we will get ten percent with randomness. 
\clearpage
Exercise 3.18 (c)
Since a complete alaphabetical list is a vailable the most apporpiate method would be random sampling since we would get a random response. 
\clearpage
Exercise 3.18 (d)
Cluster sampling every college will form a cluster and the students will be the elements. This works since the sample is already organized by college. 

\end{document}